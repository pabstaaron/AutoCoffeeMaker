
%% bare_conf.tex
%% V1.4b
%% 2015/08/26
%% by Michael Shell
%% See:
%% http://www.michaelshell.org/
%% for current contact information.
%%
%% This is a skeleton file demonstrating the use of IEEEtran.cls
%% (requires IEEEtran.cls version 1.8b or later) with an IEEE
%% conference paper.
%%
%% Support sites:
%% http://www.michaelshell.org/tex/ieeetran/
%% http://www.ctan.org/pkg/ieeetran
%% and
%% http://www.ieee.org/

%%*************************************************************************
%% Legal Notice:
%% This code is offered as-is without any warranty either expressed or
%% implied; without even the implied warranty of MERCHANTABILITY or
%% FITNESS FOR A PARTICULAR PURPOSE! 
%% User assumes all risk.
%% In no event shall the IEEE or any contributor to this code be liable for
%% any damages or losses, including, but not limited to, incidental,
%% consequential, or any other damages, resulting from the use or misuse
%% of any information contained here.
%%
%% All comments are the opinions of their respective authors and are not
%% necessarily endorsed by the IEEE.
%%
%% This work is distributed under the LaTeX Project Public License (LPPL)
%% ( http://www.latex-project.org/ ) version 1.3, and may be freely used,
%% distributed and modified. A copy of the LPPL, version 1.3, is included
%% in the base LaTeX documentation of all distributions of LaTeX released
%% 2003/12/01 or later.
%% Retain all contribution notices and credits.
%% ** Modified files should be clearly indicated as such, including  **
%% ** renaming them and changing author support contact information. **
%%*************************************************************************


% *** Authors should verify (and, if needed, correct) their LaTeX system  ***
% *** with the testflow diagnostic prior to trusting their LaTeX platform ***
% *** with production work. The IEEE's font choices and paper sizes can   ***
% *** trigger bugs that do not appear when using other class files.       ***                          ***
% The testflow support page is at:
% http://www.michaelshell.org/tex/testflow/



\documentclass[conference]{IEEEtran}

%
\ifCLASSINFOpdf
  % \usepackage[pdftex]{graphicx}
  % declare the path(s) where your graphic files are
  % \graphicspath{{../pdf/}{../jpeg/}}
  % and their extensions so you won't have to specify these with
  % every instance of \includegraphics
  % \DeclareGraphicsExtensions{.pdf,.jpeg,.png}
\else
  % or other class option (dvipsone, dvipdf, if not using dvips). graphicx
  % will default to the driver specified in the system graphics.cfg if no
  % driver is specified.
  % \usepackage[dvips]{graphicx}
  % declare the path(s) where your graphic files are
  % \graphicspath{{../eps/}}
  % and their extensions so you won't have to specify these with
  % every instance of \includegraphics
  % \DeclareGraphicsExtensions{.eps}
\fi



% correct bad hyphenation here
\hyphenation{op-tical net-works semi-conduc-tor}


\begin{document}

\title{Senior Project Proposal}


% author names and affiliations
% use a multiple column layout for up to three different
% affiliations
\author{Aaron~Pabst, Ben~Nagel, Nathan~Donaldson, Elliot~CarrLee}

% conference papers do not typically use \thanks and this command
% is locked out in conference mode. If really needed, such as for
% the acknowledgment of grants, issue a \IEEEoverridecommandlockouts
% after \documentclass

% for over three affiliations, or if they all won't fit within the width
% of the page, use this alternative format:
% 
%\author{\IEEEauthorblockN{Michael Shell\IEEEauthorrefmark{1},
%Homer Simpson\IEEEauthorrefmark{2},
%James Kirk\IEEEauthorrefmark{3}, 
%Montgomery Scott\IEEEauthorrefmark{3} and
%Eldon Tyrell\IEEEauthorrefmark{4}}
%\IEEEauthorblockA{\IEEEauthorrefmark{1}School of Electrical and Computer Engineering\\
%Georgia Institute of Technology,
%Atlanta, Georgia 30332--0250\\ Email: see http://www.michaelshell.org/contact.html}
%\IEEEauthorblockA{\IEEEauthorrefmark{2}Twentieth Century Fox, Springfield, USA\\
%Email: homer@thesimpsons.com}
%\IEEEauthorblockA{\IEEEauthorrefmark{3}Starfleet Academy, San Francisco, California 96678-2391\\
%Telephone: (800) 555--1212, Fax: (888) 555--1212}
%\IEEEauthorblockA{\IEEEauthorrefmark{4}Tyrell Inc., 123 Replicant Street, Los Angeles, California 90210--4321}}


% make the title area
\maketitle

% As a general rule, do not put math, special symbols or citations
% in the abstract
\begin{abstract}
Imagine having a machine that would allow you to make the perfect cup of coffee
right in your own home or office, without ever having to trek to the coffee
shop. A machine that allows you to specify exactly how you like your coffee
with exacting detail and optional scheduling at which point it will flawlessly
produce that beverage for you. A machine that can recommend new beverages based
on your and other users’ past preferences.

There are plenty of machines on the market that claim to be fully automatic,
but none are capable of going all the way from raw ingredients to a finished
beverage without any intervention from the user. With these
machines, the user still has to manually froth milk dispense flavoring
syrup, and mix the beverage themselves; the machine only handles the grinding,
tamping, and brewing tasks.
These machines also provide a very limited scope of control to the user, giving them
only a few options for dictating how they would like their coffee to be
produced.
Finally, most of these machines have very limited user interfaces and few have
an option for remote control from a more user friendly device, such as a
smart phone or tablet.

Our product differs by focusing on automation, personalization, and
user-friendliness above all else. Our machine will house all ingredients
internally, receive instructions wirelessly, and will have an automatic
cleaning mechanism. The minimization of human interaction allows us to have
complete control of the brewing and mixing process, allowing us to make a
consistent cup of coffee. This also allows us to operate the machine
wirelessly, allowing for streamlined UI capabilities and scheduled events.
Autonomous control allows you to monitor the machine’s use, personal
consumption habits, and ingredient consumption, while also giving the user the
ability to tweak individual settings to make a personalized and reproducible
cup of coffee. The machine itself will consist of a series of boilers,
chillers, and pumps as well as a specially designed chamber for automatically
frothing milk. These components will be driven from custom heater and chiller
control circuitry as well as an embedded linux controller. The physical device
will be backed by some remote user-interface and a database for storing
ingredient information and user data.
\end{abstract}


% What is the project motivation? 
%
% TODO - Could use a citation or two in here
% TODO - Add more background information
% TODO - There are a couple machines out there that can froth milk on their own,
% but they do so poorly
\section{Introduction} 

There are many espresso machines available for commercial and
consumer applications. The most frequent of which an average consumer will
encounter is the manual espresso machine used by their local coffee shop. These
machines require a substantial amount of training and practice to wield
effectively. The operator must master the skills of grinding the coffee, tamping
grounds, and physically pulling the espresso; tasks that are out of reach for
the average consumer to perform on their own. What's more, the operator must
manually froth milk and dispense an appropriate amount of flavoring syrup for more
complicated bevrages, such as lattes.

However, there are an increasing number of espresso machines on the market that
automate some of this process.  Most of these machines grind beans, tamp the
grounds, and pull the espresso with little intervention from the user. The user,
however, must still manually froth milk (a difficult task to master), deploy
flavoring syrup, and mix the the beverage on their own.
 
There is not presently an \emph{eleqantly} implemented solution for a fully
automated espresso machine that is easy and fast for the average consumer
to use. This is largely due to the fact that there are certain mechanisms and
control systems that would have to be created for such a device that are
non-trivial to design and implement.

\subsection{How Espresso Machines Work}
% TODO there may be some light plagerism in this section..

There are many different types of coffee makers available. Each of these systems
is unique in its own way and each has its pro's and con's. All coffee makers,
however, have one thing in common; they all must push hot water through ground
coffee beans in some way. In the case of espresso, hot water is heated to a near
boiling temprature and pressurized in some way. This hot water is then forced
through densely packed coffee grounds (known as a ``puck'') in order to produce
a thick, creamy coffee \cite{wikiespresso}.  

There are several different ways in which the water may be pressurized. One of
the most common ways this is accomplished is to simply let the water pressurize
as it heats in a sealed container and turns into steam. Once a suitable
temprature is reached, the container is unsealed and the pressurized water
passes through the puck. This approach is used in most low-end espresso machines
as it requires few mechanical components and is generally inexpensive. It also
tends to produce lower quality espresso as the water pressure is difficult to
regulate and drops as the brewing cycle progresses.

Higher end machines used in most commercial applications use an electric pump to
force the water through the grounds, allowing for tighter control of the
water pressure as well as faster brewing times.

An optional, but important, component of an espresso machine is the frothing
wand. The frothing wand is a hollow shaft of aluminium that is used to direct
high pressure steam into milk (or a milk-like product), the effect of which is
incorporating air into the milk that makes it light and foamy (or frothy, as
the name suggestes). Most modern espresso machines have a built-in frothing
mechanism. On lower end machines, the frothing wand connectes to the same boiler
that produces the brewing water. This is undesirable due to the fact that
frothing water needs to be heated to much higher tempratures than brewing water
in order to produce the necessary high pressure steam. Machines that only have a
single boiler therefore need to introduce a long delay between the brewing and
frothing cycle while the boiler switches from one task to another.

Higher-end machines will generally introduce a second boiler for producing
frothing steam. This boiler will operate at a much higher temprature than the
brewing boiler in order to produce the necessary steam pressure.

Another important task that plays a large role in
producing high quality espresso is the grinding of the beans. While this task
seems simple, there are many factors introduced in griding that can have a large
impact in the flavor and consistency of the final product. 

Coffee for espresso is usually ground to a very fine powder so that it can be
densly packed into a puck. This is subject to personal preferance, however, and some
may prefer the flavor of a coarser grind. In many modern espresso machines, the
grinder is integrated into the machine. Many baristas, however, prefer to use a
seperate grinder to give them more control over the process.

Most grinders used in commercial and upscale domestic applications are burr
grinders. Burr grinders have two vertically aligned steel plates with an air gap
between them. A hopper containing coffee beans sits above these two plates and
passes a handful of coffee beans at a time into the gap between the two plates.
The plates are roatated manually or via an electric motor to crush the beans,
which are then allowed to fall into a vessel containing the ground coffee. The
space between the blades determines the fineness of the grind.

The remainder of this document discusses the implementation of a fully
automated, web connected espresso machine.

\section{Project Overview}
% Flask Overview?
 
\section{Project Tasks}
% Just guessing here, but in an ideal world this would be the waterfall outward
% Inidividual Components working
% Components Controlled By I2C
% Components Controlled via USART Serial
% Components Controlled via Flask API Calls
% Flush out Flask buffer and all calls
%% We probably need a buffering system since the machine throughput will be low
% Flask talking to DB
% Flask talking to Android Device
% DB Pushing valuable data downstream


\section{Project Materials}
% Raspberry Pi
% Android Phone

\section{Testing Approach}
% Flask
Testing against the Flask microframework part of our project will be heavily reliant on 'mocking'
outputs and inputs of recieved calls. Flask as a microframework does not need to be tested, however
our devices reaction to GPIO eventst needs to be tested. This can be occumplished using the python
unittest.mock framework  \cite{mock} where we can create tests that will 'mock' returned data to
be able to flush out multiple tests without relying on any network or GPIO setup. 


\section{Project Demonstration}
The goal for our project demonstration is that we can show give a user a phone, let him customize a
cup of coffee, and have it make a cup of coffee without anyone user interference. Our demonstration
is meant to be simple, because our project is meant to simplify. If we can walk away from our
machine and still have it function correctly in the hands on people unfamiliar we would completed
what we set out to do.

% \begin{thebibliography}{1}
% 
% \bibitem{IEEEhowto:kopka}
% H.~Kopka and P.~W. Daly, \emph{A Guide to \LaTeX}, 3rd~ed.\hskip 1em plus
%   0.5em minus 0.4em\relax Harlow, England: Addison-Wesley, 1999.
% 
% \end{thebibliography}

\bibliography{biblio}{}
\bibliographystyle{plain}

\end{document}

\end{document}


