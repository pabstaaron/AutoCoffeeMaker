
%% bare_conf.tex
%% V1.4b
%% 2015/08/26
%% by Michael Shell
%% See:
%% http://www.michaelshell.org/
%% for current contact information.
%%
%% This is a skeleton file demonstrating the use of IEEEtran.cls
%% (requires IEEEtran.cls version 1.8b or later) with an IEEE
%% conference paper.
%%
%% Support sites:
%% http://www.michaelshell.org/tex/ieeetran/
%% http://www.ctan.org/pkg/ieeetran
%% and
%% http://www.ieee.org/

%%*************************************************************************
%% Legal Notice:
%% This code is offered as-is without any warranty either expressed or
%% implied; without even the implied warranty of MERCHANTABILITY or
%% FITNESS FOR A PARTICULAR PURPOSE! 
%% User assumes all risk.
%% In no event shall the IEEE or any contributor to this code be liable for
%% any damages or losses, including, but not limited to, incidental,
%% consequential, or any other damages, resulting from the use or misuse
%% of any information contained here.
%%
%% All comments are the opinions of their respective authors and are not
%% necessarily endorsed by the IEEE.
%%
%% This work is distributed under the LaTeX Project Public License (LPPL)
%% ( http://www.latex-project.org/ ) version 1.3, and may be freely used,
%% distributed and modified. A copy of the LPPL, version 1.3, is included
%% in the base LaTeX documentation of all distributions of LaTeX released
%% 2003/12/01 or later.
%% Retain all contribution notices and credits.
%% ** Modified files should be clearly indicated as such, including  **
%% ** renaming them and changing author support contact information. **
%%*************************************************************************


% *** Authors should verify (and, if needed, correct) their LaTeX system  ***
% *** with the testflow diagnostic prior to trusting their LaTeX platform ***
% *** with production work. The IEEE's font choices and paper sizes can   ***
% *** trigger bugs that do not appear when using other class files.       ***                          ***
% The testflow support page is at:
% http://www.michaelshell.org/tex/testflow/



\documentclass[conference]{IEEEtran}

%
\ifCLASSINFOpdf
  % \usepackage[pdftex]{graphicx}
  % declare the path(s) where your graphic files are
  % \graphicspath{{../pdf/}{../jpeg/}}
  % and their extensions so you won't have to specify these with
  % every instance of \includegraphics
  % \DeclareGraphicsExtensions{.pdf,.jpeg,.png}
\else
  % or other class option (dvipsone, dvipdf, if not using dvips). graphicx
  % will default to the driver specified in the system graphics.cfg if no
  % driver is specified.
  % \usepackage[dvips]{graphicx}
  % declare the path(s) where your graphic files are
  % \graphicspath{{../eps/}}
  % and their extensions so you won't have to specify these with
  % every instance of \includegraphics
  % \DeclareGraphicsExtensions{.eps}
\fi



% correct bad hyphenation here
\hyphenation{op-tical net-works semi-conduc-tor}


\begin{document}

\title{Senior Project Proposal}


% author names and affiliations
% use a multiple column layout for up to three different
% affiliations
\author{Aaron~Pabst, Ben~Nagel, Nathan~Donaldson, Elliot~CarrLee}

% conference papers do not typically use \thanks and this command
% is locked out in conference mode. If really needed, such as for
% the acknowledgment of grants, issue a \IEEEoverridecommandlockouts
% after \documentclass

% for over three affiliations, or if they all won't fit within the width
% of the page, use this alternative format:
% 
%\author{\IEEEauthorblockN{Michael Shell\IEEEauthorrefmark{1},
%Homer Simpson\IEEEauthorrefmark{2},
%James Kirk\IEEEauthorrefmark{3}, 
%Montgomery Scott\IEEEauthorrefmark{3} and
%Eldon Tyrell\IEEEauthorrefmark{4}}
%\IEEEauthorblockA{\IEEEauthorrefmark{1}School of Electrical and Computer Engineering\\
%Georgia Institute of Technology,
%Atlanta, Georgia 30332--0250\\ Email: see http://www.michaelshell.org/contact.html}
%\IEEEauthorblockA{\IEEEauthorrefmark{2}Twentieth Century Fox, Springfield, USA\\
%Email: homer@thesimpsons.com}
%\IEEEauthorblockA{\IEEEauthorrefmark{3}Starfleet Academy, San Francisco, California 96678-2391\\
%Telephone: (800) 555--1212, Fax: (888) 555--1212}
%\IEEEauthorblockA{\IEEEauthorrefmark{4}Tyrell Inc., 123 Replicant Street, Los Angeles, California 90210--4321}}


% make the title area
\maketitle

% As a general rule, do not put math, special symbols or citations
% in the abstract
\begin{abstract}
Imagine having a machine that would allow you to make the perfect cup of coffee
right in your own home or office, without ever having to trek to the coffee
shop. A machine that allows you to specify exactly how you like your coffee
with exacting detail and optional scheduling at which point it will flawlessly
produce that beverage for you. A machine that can recommend new beverages based
on your and other users’ past preferences.

There are plenty of machines on the market that claim to be fully automatic,
but none are capable of going all the way from raw ingredients to a finished
beverage without any intervention from the user. With these
machines, the user still has to manually froth milk dispense flavoring
syrup, and mix the beverage themselves; the machine only handles the grinding,
tamping, and brewing tasks.
These machines also provide a very limited scope of control to the user, giving them
only a few options for dictating how they would like their coffee to be
produced.
Finally, most of these machines have very limited user interfaces and few have
an option for remote control from a more user friendly device, such as a
smartphone or tablet.

Our product differs by focusing on automation, personalization, and
user-friendliness above all else. Our machine will house all ingredients
internally, receive instructions wirelessly, and will have an automatic
cleaning mechanism. The minimization of human interaction allows us to have
complete control of the brewing and mixing process, allowing us to make a
consistent cup of coffee. This also allows us to operate the machine
wirelessly, allowing for streamlined UI capabilities and scheduled events.
Autonomous control allows you to monitor the machine’s use, personal
consumption habits, and ingredient consumption, while also giving the user the
ability to tweak individual settings to make a personalized and reproducible
cup of coffee. The machine itself will consist of a series of boilers,
chillers, and pumps as well as a specially designed chamber for automatically
frothing milk. These components will be driven from custom heater and chiller
control circuitry as well as an embedded linux controller. The physical device
will be backed by some remote user-interface and a database for storing
ingredient information and user data.
\end{abstract}


% What is the project motivation? 
%
\section{Introduction}

\section{Project Overview}

\section{Project Tasks}

\section{Project Materials}

\section{Testing Approach}

\section{Project Demonstration}

% \begin{thebibliography}{1}
% 
% \bibitem{IEEEhowto:kopka}
% H.~Kopka and P.~W. Daly, \emph{A Guide to \LaTeX}, 3rd~ed.\hskip 1em plus
%   0.5em minus 0.4em\relax Harlow, England: Addison-Wesley, 1999.
% 
% \end{thebibliography}

\end{document}


